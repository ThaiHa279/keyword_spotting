\chapter*{Lời cảm ơn}
Mỗi sự thành công đều gắn liền với những sự giúp đỡ, hỗ trợ dù ít hay nhiều, dù trực  tiếp hay gián tiếp của người khác. Trong suốt khoảng thời gian từ những bước chân đầu tiên  đến giảng đường đại học đến ngày hôm nay, em đã nhận được rất nhiều sự qua tâm, giúp đỡ của quý thầy cô, gia đình, bạn bè.

Em xin gửi lời cảm ơn chân thành nhất đến thầy Phạm Nguyên Hoàng, người hướng dẫn Niên luận của tôi, vì đã dành thời gian, kiến thức và kinh nghiệm quý báu của thầy để hỗ trợ em hoàn thành Niên luận này. Em cũng xin cảm ơn các giáo viên trong khoa Công nghệ thông tin và Truyền thông vì đã đưa ra những nhận xét, góp ý và khuyến khích giúp tôi cải thiện Niên luận của mình.

Với điều kiện thời gian cũng như kinh nghiệm còn hạn chế, bài báo cáo này không thể tránh được những thiếu sót. Em rất mong nhận được sự chỉ bảo, đóng góp ý kiến của các quý  thầy cô để em có điều kiện bổ sung, nâng cao ý thức của mình. 

Em xin chân thành cảm ơn! \par
\begin{table}[!ht]\centering
    \begin{tabular}{p{.45\linewidth}p{.5\linewidth}}
        &\centering
        \textit{Cần Thơ, ngày \the\day{} tháng \the\month{} năm \the\year{}}\par
        \textbf{Sinh viên thực hiện}\par
        \vspace{3cm}
        \textbf{Vũ Thái Hà}\par
    \end{tabular}
\end{table}
\cleardoublepage

\chapter*{Nhận xét của giảng viên}
\foreach \n in {0,...,27}{
    \noindent\hdashrule{\textwidth}{1pt}{1mm} % để vẽ một đường kẻ chấm chiều dài bằng chiều rộng của trang
}
\cleardoublepage

\chapter*{Tóm tắt}
Niên luận này nghiên cứu về vấn đề nhận dạng từ khóa trong câu (KWS), một bài toán quan trọng trong xử lý ngôn ngữ tự nhiên (NLP). KWS là việc tìm ra những từ hoặc cụm từ có ý nghĩa quan trọng hoặc đặc trưng cho nội dung của một câu. KWS có nhiều ứng dụng thực tế, như tìm kiếm thông tin, phân loại văn bản, tóm tắt văn bản, phân tích cảm xúc, v.v. 

Niên luận này đề xuất một phương pháp để giải quyết bài toán KWS, sử dụng kỹ thuật học sâu (deep learning). Phương pháp này gồm hai bước chính: phát hiện từ và nhận diện từ. Bước phát hiện từ  sử dụng chỉ số RMSE để lọc tiếng ồn và phát hiện một từ trong câu. Bước nhận diện từ sử dụng kết hợp mô hình CNN và LSTM để dự đoán xác suất từ đã nhận dạng. Phương pháp này được huấn luyện trên một tập dữ liệu speech commands version 2 của tensorflow. 

Niên luận này đã đề xuất một phương pháp triển khai một mô hình học sâu lên một ứng dụng web. Niên luận này cũng đề cập đến một số hướng phát triển tiếp theo của phương pháp đề xuất, như mở rộng cho nhiều ngôn ngữ, sử dụng các mô hình học sâu khác, v.v.  \par
\cleardoublepage

\chapter*{Abstract}
This thesis studies the problem of keyword spotting in sentences (KWS), an important task in natural language processing (NLP). KWS is the process of finding words or phrases that have significant or distinctive meanings for the content of a sentence. KWS has many practical applications, such as information retrieval, text classification, text summarization, sentiment analysis, etc.

This thesis proposes a method to solve the KWS problem, using deep learning techniques. The method consists of two main steps: word detection and word recognition. The word detection step uses the RMSE index to filter noise and detect a word in a sentence. The word recognition step uses a combination of CNN and LSTM models to predict the probability of the detected word. The method is trained on a speech commands version 2 dataset from tensorflow.

This thesis also proposes a method to implement a deep learning model on a web application. This thesis also discusses some future directions of the proposed method, such as extending to multiple languages, using different deep learning models, etc.\par
\cleardoublepage